% header.tex

% \documentclass{article}
\usepackage{fancyhdr}
\usepackage{listings}
\usepackage{tcolorbox}
\usepackage{xeCJK}  % for handling chinese, japanese, korean
\usepackage{hyperref}
% 
% \usepackage{arxiv}
% \usepackage[utf8]{inputenc} % allow utf-8 input
% \usepackage[T1]{fontenc}    % use 8-bit T1 fonts
% \usepackage{url}            % simple URL typesetting
% \usepackage{booktabs}       % professional-quality tables
% \usepackage{amsfonts}       % blackboard math symbols
% \usepackage{nicefrac}       % compact symbols for 1/2, etc.
% \usepackage{microtype}      % microtypography
% \usepackage{graphicx}
% \usepackage{natbib}
% \usepackage{doi}
% \usepackage{float} %
% \usepackage{amsmath} %


\setCJKmainfont{標楷體}  % set main font

\pagestyle{fancy}
\fancyhf{}  % clears the header and footer
\renewcommand{\footrulewidth}{0.6pt}  % Change the thickness of the line (0.4pt is the default)
\renewcommand{\headrulewidth}{0.6pt}

% helps to wrap up codes and texts
% unfortunately, the output will be very ugly
% Note: "pandoc_args: --listings" is a must in the .Rmd file
%\lstset{
    %breaklines = true,
%}

% Add text above/below the header/footer line
% \lhead{\textit{R Note}}
% \lhead{\textit{\nouppercase{\leftmark}}: \textit{\nouppercase{\rightmark}}  % Show section name on the top left of every page
\lhead{\textit{\nouppercase{\leftmark}: \nouppercase{\rightmark}}}  % Show section name on the top left of every page
\rhead{\thepage}  % Place the page number on the right side of the footer
\lfoot{\hyperref[sec:ToC]{\textcolor{blue}{Back to \textbf{Table of Contents}}}}  % date
\rfoot{\textit{\copyright\ Created by Potumas Liu}}  % Replace with your desired text

\newcommand{\sectionbreak}{\clearpage}