% Options for packages loaded elsewhere
\PassOptionsToPackage{unicode}{hyperref}
\PassOptionsToPackage{hyphens}{url}
%
\documentclass[
  12pt,
]{article}
\usepackage{amsmath,amssymb}
\usepackage{iftex}
\ifPDFTeX
  \usepackage[T1]{fontenc}
  \usepackage[utf8]{inputenc}
  \usepackage{textcomp} % provide euro and other symbols
\else % if luatex or xetex
  \usepackage{unicode-math} % this also loads fontspec
  \defaultfontfeatures{Scale=MatchLowercase}
  \defaultfontfeatures[\rmfamily]{Ligatures=TeX,Scale=1}
\fi
\usepackage{lmodern}
\ifPDFTeX\else
  % xetex/luatex font selection
\fi
% Use upquote if available, for straight quotes in verbatim environments
\IfFileExists{upquote.sty}{\usepackage{upquote}}{}
\IfFileExists{microtype.sty}{% use microtype if available
  \usepackage[]{microtype}
  \UseMicrotypeSet[protrusion]{basicmath} % disable protrusion for tt fonts
}{}
\makeatletter
\@ifundefined{KOMAClassName}{% if non-KOMA class
  \IfFileExists{parskip.sty}{%
    \usepackage{parskip}
  }{% else
    \setlength{\parindent}{0pt}
    \setlength{\parskip}{6pt plus 2pt minus 1pt}}
}{% if KOMA class
  \KOMAoptions{parskip=half}}
\makeatother
\usepackage{xcolor}
\usepackage[margin=1in]{geometry}
\usepackage{graphicx}
\makeatletter
\def\maxwidth{\ifdim\Gin@nat@width>\linewidth\linewidth\else\Gin@nat@width\fi}
\def\maxheight{\ifdim\Gin@nat@height>\textheight\textheight\else\Gin@nat@height\fi}
\makeatother
% Scale images if necessary, so that they will not overflow the page
% margins by default, and it is still possible to overwrite the defaults
% using explicit options in \includegraphics[width, height, ...]{}
\setkeys{Gin}{width=\maxwidth,height=\maxheight,keepaspectratio}
% Set default figure placement to htbp
\makeatletter
\def\fps@figure{htbp}
\makeatother
\setlength{\emergencystretch}{3em} % prevent overfull lines
\providecommand{\tightlist}{%
  \setlength{\itemsep}{0pt}\setlength{\parskip}{0pt}}
\setcounter{secnumdepth}{-\maxdimen} % remove section numbering
% header.tex

% \documentclass{article}
\usepackage{fancyhdr}
\usepackage{listings}
\usepackage{tcolorbox}
\usepackage{xeCJK}  % for handling chinese, japanese, korean
\usepackage{hyperref}
% 
% \usepackage{arxiv}
% \usepackage[utf8]{inputenc} % allow utf-8 input
% \usepackage[T1]{fontenc}    % use 8-bit T1 fonts
% \usepackage{url}            % simple URL typesetting
% \usepackage{booktabs}       % professional-quality tables
% \usepackage{amsfonts}       % blackboard math symbols
% \usepackage{nicefrac}       % compact symbols for 1/2, etc.
% \usepackage{microtype}      % microtypography
% \usepackage{graphicx}
% \usepackage{natbib}
% \usepackage{doi}
% \usepackage{float} %
% \usepackage{amsmath} %


\setCJKmainfont{標楷體}  % set main font

\pagestyle{fancy}
\fancyhf{}  % clears the header and footer
\renewcommand{\footrulewidth}{0.6pt}  % Change the thickness of the line (0.4pt is the default)
\renewcommand{\headrulewidth}{0.6pt}

% helps to wrap up codes and texts
% unfortunately, the output will be very ugly
% Note: "pandoc_args: --listings" is a must in the .Rmd file
%\lstset{
    %breaklines = true,
%}

% Add text above/below the header/footer line
% \lhead{\textit{R Note}}
% \lhead{\textit{\nouppercase{\leftmark}}: \textit{\nouppercase{\rightmark}}  % Show section name on the top left of every page
\lhead{\textit{\nouppercase{\leftmark}: \nouppercase{\rightmark}}}  % Show section name on the top left of every page
\rhead{\thepage}  % Place the page number on the right side of the footer
\lfoot{\hyperref[sec:ToC]{\textcolor{blue}{Back to \textbf{Table of Contents}}}}  % date
\rfoot{\textit{\copyright\ Created by Potumas Liu}}  % Replace with your desired text

\newcommand{\sectionbreak}{\clearpage}
\renewcommand{\and}{\\}
\ifLuaTeX
  \usepackage{selnolig}  % disable illegal ligatures
\fi
\usepackage{bookmark}
\IfFileExists{xurl.sty}{\usepackage{xurl}}{} % add URL line breaks if available
\urlstyle{same}
\hypersetup{
  pdftitle={R Note},
  pdfauthor={Chih-Tse Liu},
  hidelinks,
  pdfcreator={LaTeX via pandoc}}

\title{R Note}
\author{Chih-Tse Liu}
\date{08 June, 2024}

\begin{document}
\maketitle

\section{Table of Contents}
\label{sec:ToC}
\begin{enumerate}
\item \hyperref[sec:BRS]{Basic R Syntax}
  \begin{enumerate}
  \item \hyperref[sec:BRS-conditions]{Conditions}
  \item \hyperref[sec:BRS-loops]{Loops}
  \item \hyperref[sec:BRS-function]{Function}
  \item \hyperref[sec:BRS-]{}
  \item \hyperref[sec:BRS-]{}
  \item \hyperref[sec:BRS-]{}
  \item \hyperref[sec:BRS-]{}
  \end{enumerate}
\item \hyperref[sec:VAR]{Variable Type}
  \begin{enumerate}
  \item \hyperref[sec:VAR-numeric]{Numeric}
  \item \hyperref[sec:VAR-integer]{Integer}
  \item \hyperref[sec:VAR-character]{Character}
  \item \hyperref[sec:VAR-factor]{Factor}
  \item \hyperref[sec:VAR-logical]{Logical}
  \item \hyperref[sec:VAR-]{}
  \end{enumerate}
\item \hyperref[sec:DS]{Data Structure}
  \begin{enumerate}
  \item \hyperref[sec:DS-vector]{Vector}
  \item \hyperref[sec:DS-array]{Array}
  \item \hyperref[sec:DS-dataframe]{Data Frame}
  \item \hyperref[sec:DS-list]{List}
  \item \hyperref[sec:DS-matrix]{Matrix}
  \item \hyperref[sec:DS-]{}
  \item \hyperref[sec:DS-]{}
  \item \hyperref[sec:DS-]{}
  \item \hyperref[sec:DS-]{}
  \end{enumerate}
\item \hyperref[sec:Fig]{Figures}
  \begin{enumerate}
  \item \hyperref[sec:Fig-]{Scatter Plot}
  \item \hyperref[sec:Fig-]{Box Plot}
  \item \hyperref[sec:Fig-]{}
  \end{enumerate}
\item \hyperref[sec:SM]{String Manipulation}
  \begin{enumerate}
  \item \hyperref[sec:SM-]{}
  \item \hyperref[sec:SM-]{}
  \item \hyperref[sec:SM-]{}
  \end{enumerate}
\item \hyperref[sec:Pkg]{Important Packages}
  \begin{enumerate}
  \item \hyperref[sec:Pkg-]{}
  \item \hyperref[sec:Pkg-]{}
  \item \hyperref[sec:Pkg-]{}
  \end{enumerate}
\item \hyperref[sec:Misc]{Misc}
  \begin{enumerate}
  \item \hyperref[sec:Misc-]{}
  \item \hyperref[sec:Misc-]{}
  \item \hyperref[sec:Misc-]{}
  \end{enumerate}
\end{enumerate}

\newpage
\section{Basic R Syntax}
\label{sec:BRS}

\newpage
\section{Variable Type}
\label{sec:VAR}

\newpage
\section{Data Structure}
\label{sec:DS}

\newpage
\section{Figure}
\label{sec:Fig}

\newpage
\section{String Manipulation}
\label{sec:SM}

\newpage
\section{Important Packages}
\label{sec:Pkg}

\newpage
\section{Misc}
\label{sec:Misc}

\end{document}
