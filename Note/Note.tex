% Options for packages loaded elsewhere
\PassOptionsToPackage{unicode}{hyperref}
\PassOptionsToPackage{hyphens}{url}
%
\documentclass[
  12pt,
]{article}
\usepackage{amsmath,amssymb}
\usepackage{iftex}
\ifPDFTeX
  \usepackage[T1]{fontenc}
  \usepackage[utf8]{inputenc}
  \usepackage{textcomp} % provide euro and other symbols
\else % if luatex or xetex
  \usepackage{unicode-math} % this also loads fontspec
  \defaultfontfeatures{Scale=MatchLowercase}
  \defaultfontfeatures[\rmfamily]{Ligatures=TeX,Scale=1}
\fi
\usepackage{lmodern}
\ifPDFTeX\else
  % xetex/luatex font selection
\fi
% Use upquote if available, for straight quotes in verbatim environments
\IfFileExists{upquote.sty}{\usepackage{upquote}}{}
\IfFileExists{microtype.sty}{% use microtype if available
  \usepackage[]{microtype}
  \UseMicrotypeSet[protrusion]{basicmath} % disable protrusion for tt fonts
}{}
\makeatletter
\@ifundefined{KOMAClassName}{% if non-KOMA class
  \IfFileExists{parskip.sty}{%
    \usepackage{parskip}
  }{% else
    \setlength{\parindent}{0pt}
    \setlength{\parskip}{6pt plus 2pt minus 1pt}}
}{% if KOMA class
  \KOMAoptions{parskip=half}}
\makeatother
\usepackage{xcolor}
\usepackage[margin=1in]{geometry}
\usepackage{color}
\usepackage{fancyvrb}
\newcommand{\VerbBar}{|}
\newcommand{\VERB}{\Verb[commandchars=\\\{\}]}
\DefineVerbatimEnvironment{Highlighting}{Verbatim}{commandchars=\\\{\}}
% Add ',fontsize=\small' for more characters per line
\usepackage{framed}
\definecolor{shadecolor}{RGB}{248,248,248}
\newenvironment{Shaded}{\begin{snugshade}}{\end{snugshade}}
\newcommand{\AlertTok}[1]{\textcolor[rgb]{0.94,0.16,0.16}{#1}}
\newcommand{\AnnotationTok}[1]{\textcolor[rgb]{0.56,0.35,0.01}{\textbf{\textit{#1}}}}
\newcommand{\AttributeTok}[1]{\textcolor[rgb]{0.13,0.29,0.53}{#1}}
\newcommand{\BaseNTok}[1]{\textcolor[rgb]{0.00,0.00,0.81}{#1}}
\newcommand{\BuiltInTok}[1]{#1}
\newcommand{\CharTok}[1]{\textcolor[rgb]{0.31,0.60,0.02}{#1}}
\newcommand{\CommentTok}[1]{\textcolor[rgb]{0.56,0.35,0.01}{\textit{#1}}}
\newcommand{\CommentVarTok}[1]{\textcolor[rgb]{0.56,0.35,0.01}{\textbf{\textit{#1}}}}
\newcommand{\ConstantTok}[1]{\textcolor[rgb]{0.56,0.35,0.01}{#1}}
\newcommand{\ControlFlowTok}[1]{\textcolor[rgb]{0.13,0.29,0.53}{\textbf{#1}}}
\newcommand{\DataTypeTok}[1]{\textcolor[rgb]{0.13,0.29,0.53}{#1}}
\newcommand{\DecValTok}[1]{\textcolor[rgb]{0.00,0.00,0.81}{#1}}
\newcommand{\DocumentationTok}[1]{\textcolor[rgb]{0.56,0.35,0.01}{\textbf{\textit{#1}}}}
\newcommand{\ErrorTok}[1]{\textcolor[rgb]{0.64,0.00,0.00}{\textbf{#1}}}
\newcommand{\ExtensionTok}[1]{#1}
\newcommand{\FloatTok}[1]{\textcolor[rgb]{0.00,0.00,0.81}{#1}}
\newcommand{\FunctionTok}[1]{\textcolor[rgb]{0.13,0.29,0.53}{\textbf{#1}}}
\newcommand{\ImportTok}[1]{#1}
\newcommand{\InformationTok}[1]{\textcolor[rgb]{0.56,0.35,0.01}{\textbf{\textit{#1}}}}
\newcommand{\KeywordTok}[1]{\textcolor[rgb]{0.13,0.29,0.53}{\textbf{#1}}}
\newcommand{\NormalTok}[1]{#1}
\newcommand{\OperatorTok}[1]{\textcolor[rgb]{0.81,0.36,0.00}{\textbf{#1}}}
\newcommand{\OtherTok}[1]{\textcolor[rgb]{0.56,0.35,0.01}{#1}}
\newcommand{\PreprocessorTok}[1]{\textcolor[rgb]{0.56,0.35,0.01}{\textit{#1}}}
\newcommand{\RegionMarkerTok}[1]{#1}
\newcommand{\SpecialCharTok}[1]{\textcolor[rgb]{0.81,0.36,0.00}{\textbf{#1}}}
\newcommand{\SpecialStringTok}[1]{\textcolor[rgb]{0.31,0.60,0.02}{#1}}
\newcommand{\StringTok}[1]{\textcolor[rgb]{0.31,0.60,0.02}{#1}}
\newcommand{\VariableTok}[1]{\textcolor[rgb]{0.00,0.00,0.00}{#1}}
\newcommand{\VerbatimStringTok}[1]{\textcolor[rgb]{0.31,0.60,0.02}{#1}}
\newcommand{\WarningTok}[1]{\textcolor[rgb]{0.56,0.35,0.01}{\textbf{\textit{#1}}}}
\usepackage{graphicx}
\makeatletter
\def\maxwidth{\ifdim\Gin@nat@width>\linewidth\linewidth\else\Gin@nat@width\fi}
\def\maxheight{\ifdim\Gin@nat@height>\textheight\textheight\else\Gin@nat@height\fi}
\makeatother
% Scale images if necessary, so that they will not overflow the page
% margins by default, and it is still possible to overwrite the defaults
% using explicit options in \includegraphics[width, height, ...]{}
\setkeys{Gin}{width=\maxwidth,height=\maxheight,keepaspectratio}
% Set default figure placement to htbp
\makeatletter
\def\fps@figure{htbp}
\makeatother
\setlength{\emergencystretch}{3em} % prevent overfull lines
\providecommand{\tightlist}{%
  \setlength{\itemsep}{0pt}\setlength{\parskip}{0pt}}
\setcounter{secnumdepth}{-\maxdimen} % remove section numbering
% header.tex

% \documentclass{article}
\usepackage{fancyhdr}
\usepackage{listings}
\usepackage{tcolorbox}
\usepackage{xeCJK}  % for handling chinese, japanese, korean
\usepackage{hyperref}
% 
% \usepackage{arxiv}
% \usepackage[utf8]{inputenc} % allow utf-8 input
% \usepackage[T1]{fontenc}    % use 8-bit T1 fonts
% \usepackage{url}            % simple URL typesetting
% \usepackage{booktabs}       % professional-quality tables
% \usepackage{amsfonts}       % blackboard math symbols
% \usepackage{nicefrac}       % compact symbols for 1/2, etc.
% \usepackage{microtype}      % microtypography
% \usepackage{graphicx}
% \usepackage{natbib}
% \usepackage{doi}
% \usepackage{float} %
% \usepackage{amsmath} %


\setCJKmainfont{標楷體}  % set main font

\pagestyle{fancy}
\fancyhf{}  % clears the header and footer
\renewcommand{\footrulewidth}{0.6pt}  % Change the thickness of the line (0.4pt is the default)
\renewcommand{\headrulewidth}{0.6pt}

% helps to wrap up codes and texts
% unfortunately, the output will be very ugly
% Note: "pandoc_args: --listings" is a must in the .Rmd file
%\lstset{
    %breaklines = true,
%}

% Add text above/below the header/footer line
% \lhead{\textit{R Note}}
% \lhead{\textit{\nouppercase{\leftmark}}: \textit{\nouppercase{\rightmark}}  % Show section name on the top left of every page
\lhead{\textit{\nouppercase{\leftmark}: \nouppercase{\rightmark}}}  % Show section name on the top left of every page
\rhead{\thepage}  % Place the page number on the right side of the footer
\lfoot{\hyperref[sec:ToC]{\textcolor{blue}{Back to \textbf{Table of Contents}}}}  % date
\rfoot{\textit{\copyright\ Created by Potumas Liu}}  % Replace with your desired text

\newcommand{\sectionbreak}{\clearpage}
\renewcommand{\and}{\\}
\ifLuaTeX
  \usepackage{selnolig}  % disable illegal ligatures
\fi
\usepackage{bookmark}
\IfFileExists{xurl.sty}{\usepackage{xurl}}{} % add URL line breaks if available
\urlstyle{same}
\hypersetup{
  pdftitle={R Note},
  pdfauthor={Chih-Tse Liu},
  hidelinks,
  pdfcreator={LaTeX via pandoc}}

\title{R Note}
\author{Chih-Tse Liu}
\date{21 June, 2024}

\begin{document}
\maketitle

\section{Table of Contents}
\label{sec:ToC}
\begin{enumerate}
\item \hyperref[sec:BRS]{Basic R Syntax}
  \begin{enumerate}
  \item \hyperref[sec:BRS-basics]{Basics}
  \item \hyperref[sec:BRS-math]{Math}
  \item \hyperref[sec:BRS-conditions]{Conditions}
  \item \hyperref[sec:BRS-loops]{Loops}
  \item \hyperref[sec:BRS-function]{Function}
  \item \hyperref[sec:BRS-]{}
  \item \hyperref[sec:BRS-]{}
  \end{enumerate}
\item \hyperref[sec:VAR]{Data Type}
  \begin{enumerate}
  \item \hyperref[sec:VAR-numeric]{Numeric}
  \item \hyperref[sec:VAR-integer]{Integer}
  \item \hyperref[sec:VAR-character]{Character}
  \item \hyperref[sec:VAR-logical]{Logical}
  \item \hyperref[sec:VAR-factor]{Factor}
  \item \hyperref[sec:VAR-complex]{Complex}
  \item \hyperref[sec:VAR-]{}
  \end{enumerate}
\item \hyperref[sec:DS]{Data Structure}
  \begin{enumerate}
  \item \hyperref[sec:DS-vector]{Vector}
  \item \hyperref[sec:DS-array]{Array}
  \item \hyperref[sec:DS-dataframe]{Data Frame}
  \item \hyperref[sec:DS-list]{List}
  \item \hyperref[sec:DS-matrix]{Matrix}
  \item \hyperref[sec:DS-]{}
  \item \hyperref[sec:DS-]{}
  \item \hyperref[sec:DS-]{}
  \item \hyperref[sec:DS-]{}
  \end{enumerate}
\item \hyperref[sec:Fig]{Figures}
  \begin{enumerate}
  \item \hyperref[sec:Fig-]{Scatter Plot}
  \item \hyperref[sec:Fig-]{Box Plot}
  \item \hyperref[sec:Fig-]{}
  \end{enumerate}
\item \hyperref[sec:SA]{Statistical Analysis}
  \begin{enumerate}
  \item \hyperref[sec:SA-EDA]{Exploratory Data Analysis}
  \item \hyperref[sec:SA-LR]{Linear Regression}
  \item \hyperref[sec:SA-PCA]{Principal Component Analysis}
  \item \hyperref[sec:SA-FA]{Factor Analysis}
  \item \hyperref[sec:SA-CA]{Clustering Analysis}
  \item \hyperref[sec:SA-DA]{Discriminant Analysis}
  \item \hyperref[sec:SA-SEM]{Structural Equation Modelling}
  \item \hyperref[sec:SA-]{}
  \end{enumerate}
\item \hyperref[sec:SM]{String Manipulation}
  \begin{enumerate}
  \item \hyperref[sec:SM-]{}
  \item \hyperref[sec:SM-]{}
  \item \hyperref[sec:SM-]{}
  \end{enumerate}
\item \hyperref[sec:Pkg]{Packages}
  \begin{enumerate}
  \item \hyperref[sec:Pkg-]{}
  \item \hyperref[sec:Pkg-]{}
  \item \hyperref[sec:Pkg-]{}
  \end{enumerate}
\item \hyperref[sec:Misc]{Misc}
  \begin{enumerate}
  \item \hyperref[sec:Misc-]{}
  \item \hyperref[sec:Misc-]{}
  \item \hyperref[sec:Misc-]{}
  \end{enumerate}
\end{enumerate}

\newpage
\section{Basic R Syntax}
\label{sec:BRS}
\subsection{Basics}
\label{sec:BRS-basics}

Comments start with a \textbf{\#}. Texts are surrounded by quotes
\textbf{'text'} or double quotes \textbf{"text"}. Numbers are recognized
directly.

\begin{Shaded}
\begin{Highlighting}[]
\CommentTok{\# This is a comment.}
\StringTok{"Hello World"}  \CommentTok{\# text}
\DecValTok{5}  \CommentTok{\# number}
\end{Highlighting}
\end{Shaded}

\textit{print()} is not a must for printing, only when the code is
executed within a function or loop. (But it is good practice to specify
it.) \textit{cat()} functions as \textit{cout()} in C++, which
`\textbackslash n' is usually necessary.

\begin{Shaded}
\begin{Highlighting}[]
\StringTok{"Hi, my name is Austin."}
\end{Highlighting}
\end{Shaded}

\begin{verbatim}
## [1] "Hi, my name is Austin."
\end{verbatim}

\begin{Shaded}
\begin{Highlighting}[]
\DecValTok{100}
\end{Highlighting}
\end{Shaded}

\begin{verbatim}
## [1] 100
\end{verbatim}

\begin{Shaded}
\begin{Highlighting}[]
\FunctionTok{print}\NormalTok{(}\StringTok{"Hi, my name is Austin."}\NormalTok{)}
\end{Highlighting}
\end{Shaded}

\begin{verbatim}
## [1] "Hi, my name is Austin."
\end{verbatim}

\begin{Shaded}
\begin{Highlighting}[]
\FunctionTok{cat}\NormalTok{(}\StringTok{"Hi, my name is Austin."}\NormalTok{, }\DecValTok{100}\NormalTok{, }\StringTok{"}\SpecialCharTok{\textbackslash{}n}\StringTok{"}\NormalTok{)}
\end{Highlighting}
\end{Shaded}

\begin{verbatim}
## Hi, my name is Austin. 100
\end{verbatim}

\textit{paste()} can concatenate elements and returns a string. (`+'
only works for numbers, not for strings!)

\begin{Shaded}
\begin{Highlighting}[]
\NormalTok{text1 }\OtherTok{=} \FunctionTok{paste}\NormalTok{(}\StringTok{"Austin"}\NormalTok{, }\StringTok{"Good."}\NormalTok{)}
\NormalTok{text1}
\end{Highlighting}
\end{Shaded}

\begin{verbatim}
## [1] "Austin Good."
\end{verbatim}

\begin{Shaded}
\begin{Highlighting}[]
\CommentTok{\# \textquotesingle{}sep\textquotesingle{} sets the character added between elements, default is a single space}
\NormalTok{text2 }\OtherTok{=} \FunctionTok{paste}\NormalTok{(}\StringTok{"Austin is "}\NormalTok{, }\DecValTok{20}\NormalTok{, }\StringTok{" years old."}\NormalTok{, }\AttributeTok{sep =} \StringTok{""}\NormalTok{)}
\NormalTok{text2}
\end{Highlighting}
\end{Shaded}

\begin{verbatim}
## [1] "Austin is 20 years old."
\end{verbatim}

Variables are created once initialized. \textbf{'<-'} is preferred over
\textbf{'='}. Multiple assignments are available.

\begin{Shaded}
\begin{Highlighting}[]
\NormalTok{x }\OtherTok{\textless{}{-}} \StringTok{"Austin"}
\NormalTok{height }\OtherTok{\textless{}{-}} \DecValTok{40}
\NormalTok{a }\OtherTok{\textless{}{-}}\NormalTok{ b }\OtherTok{\textless{}{-}}\NormalTok{ c }\OtherTok{\textless{}{-}} \StringTok{"happy"}
\FunctionTok{paste}\NormalTok{(x, }\StringTok{"is"}\NormalTok{, a, b, c, }\StringTok{"\textasciitilde{}"}\NormalTok{)}
\end{Highlighting}
\end{Shaded}

\begin{verbatim}
## [1] "Austin is happy happy happy ~"
\end{verbatim}

\textbf{<<-} is the \textbf{global operator} in R.

\begin{Shaded}
\begin{Highlighting}[]
\NormalTok{var.g }\OtherTok{\textless{}\textless{}{-}} \DecValTok{33}
\end{Highlighting}
\end{Shaded}

Assignment operators work both ways.

\begin{Shaded}
\begin{Highlighting}[]
\StringTok{"Austin"} \OtherTok{{-}\textgreater{}}\NormalTok{ y}
\DecValTok{33} \OtherTok{{-}\textgreater{}\textgreater{}}\NormalTok{ Var}
\FunctionTok{cat}\NormalTok{(y, Var)}
\end{Highlighting}
\end{Shaded}

\begin{verbatim}
## Austin 33
\end{verbatim}

Variable names can contain characters(\textbf{a\~z}),
digits(\textbf{0\~9}), underscores(\textbf{\_}) and
periods(\textbf{\.}). They can only start with characters or
\textbf{'.'}, while the latter can't be followed directly by a digit. In
R, variable names \textbf{are case sensitive}.

\begin{Shaded}
\begin{Highlighting}[]
\NormalTok{x    }\OtherTok{\textless{}{-}} \DecValTok{1}
\NormalTok{x}\FloatTok{.1}  \OtherTok{\textless{}{-}} \DecValTok{1}
\NormalTok{x1.\_ }\OtherTok{\textless{}{-}} \DecValTok{1}
\NormalTok{.\_1  }\OtherTok{\textless{}{-}} \DecValTok{1}
\NormalTok{.x1  }\OtherTok{\textless{}{-}} \DecValTok{1}
\NormalTok{.\_.  }\OtherTok{\textless{}{-}} \DecValTok{1}
\NormalTok{..   }\OtherTok{\textless{}{-}} \DecValTok{1}
\end{Highlighting}
\end{Shaded}

\newpage
\subsection{Math}
\label{sec:BRS-math}

The followings are built-in arithmetic operators. \newline Addition(+),
subtraction(-), multiplication(*), division(/) and exponent(\^{}). The
operation is conducted from left to right, with exponent(\^{}) being
processed first, multiplication(*) and division(/) the second, and
addition(+) and subtraction(-) the last.

\begin{Shaded}
\begin{Highlighting}[]
\DecValTok{10} \SpecialCharTok{+} \DecValTok{6} \SpecialCharTok{*} \DecValTok{10} \SpecialCharTok{{-}} \DecValTok{6} \SpecialCharTok{/} \DecValTok{10} \SpecialCharTok{*} \DecValTok{6} \SpecialCharTok{\^{}} \DecValTok{3}
\end{Highlighting}
\end{Shaded}

\begin{verbatim}
## [1] -59.6
\end{verbatim}

The remainder/modulus(\%\%) and the quotient(\%/\%) are also built-in
methods.

\begin{Shaded}
\begin{Highlighting}[]
\DecValTok{10} \SpecialCharTok{\%\%} \DecValTok{6}
\end{Highlighting}
\end{Shaded}

\begin{verbatim}
## [1] 4
\end{verbatim}

\begin{Shaded}
\begin{Highlighting}[]
\DecValTok{10} \SpecialCharTok{\%/\%} \DecValTok{6}
\end{Highlighting}
\end{Shaded}

\begin{verbatim}
## [1] 1
\end{verbatim}

Common mathematical functions.

\begin{Shaded}
\begin{Highlighting}[]
\FunctionTok{max}\NormalTok{(}\DecValTok{1}\SpecialCharTok{:}\DecValTok{10}\NormalTok{)       }\CommentTok{\# returns the maximum number}
\end{Highlighting}
\end{Shaded}

\begin{verbatim}
## [1] 10
\end{verbatim}

\begin{Shaded}
\begin{Highlighting}[]
\FunctionTok{min}\NormalTok{(}\DecValTok{1}\SpecialCharTok{:}\DecValTok{10}\NormalTok{)       }\CommentTok{\# returns the minimum number}
\end{Highlighting}
\end{Shaded}

\begin{verbatim}
## [1] 1
\end{verbatim}

\begin{Shaded}
\begin{Highlighting}[]
\FunctionTok{sqrt}\NormalTok{(}\DecValTok{9}\NormalTok{)         }\CommentTok{\# returns the square root}
\end{Highlighting}
\end{Shaded}

\begin{verbatim}
## [1] 3
\end{verbatim}

\begin{Shaded}
\begin{Highlighting}[]
\FunctionTok{sqrt}\NormalTok{(}\FloatTok{8.5}\NormalTok{)       }\CommentTok{\# float input can be calculated as well }
\end{Highlighting}
\end{Shaded}

\begin{verbatim}
## [1] 2.915476
\end{verbatim}

\begin{Shaded}
\begin{Highlighting}[]
\FunctionTok{abs}\NormalTok{(}\SpecialCharTok{{-}}\DecValTok{1}\NormalTok{)         }\CommentTok{\# returns the absolute value}
\end{Highlighting}
\end{Shaded}

\begin{verbatim}
## [1] 1
\end{verbatim}

\begin{Shaded}
\begin{Highlighting}[]
\FunctionTok{ceiling}\NormalTok{(}\FloatTok{2.8}\NormalTok{)    }\CommentTok{\# returns the closest greater integer}
\end{Highlighting}
\end{Shaded}

\begin{verbatim}
## [1] 3
\end{verbatim}

\begin{Shaded}
\begin{Highlighting}[]
\FunctionTok{floor}\NormalTok{(}\FloatTok{2.8}\NormalTok{)      }\CommentTok{\# returns the closest smaller integer}
\end{Highlighting}
\end{Shaded}

\begin{verbatim}
## [1] 2
\end{verbatim}

\begin{Shaded}
\begin{Highlighting}[]
\FunctionTok{round}\NormalTok{(}\FloatTok{2.123}\NormalTok{, }\DecValTok{2}\NormalTok{) }\CommentTok{\# rounds to some decimal point, default is 0}
\end{Highlighting}
\end{Shaded}

\begin{verbatim}
## [1] 2.12
\end{verbatim}

\newpage
\subsection{Conditions}
\label{sec:BRS-condition}

If statements \textit{if()} must be written in brackets. Curly brackets
\{\} are used to define the scope of the code. \textit{else if()},
following \textit{if()}, is for handling multiple conditions, while
\textit{else()} catches all other conditions. Nested if statements are
legal for R.

\begin{Shaded}
\begin{Highlighting}[]
\ControlFlowTok{if}\NormalTok{ (}\DecValTok{1} \SpecialCharTok{==} \DecValTok{2}\NormalTok{) \{}
  \FunctionTok{cat}\NormalTok{(}\StringTok{"1 is equal to 2.}\SpecialCharTok{\textbackslash{}n}\StringTok{"}\NormalTok{)}
\NormalTok{\} }\ControlFlowTok{else} \ControlFlowTok{if}\NormalTok{ (}\DecValTok{1} \SpecialCharTok{\textgreater{}} \DecValTok{2}\NormalTok{) \{}
  \FunctionTok{cat}\NormalTok{(}\StringTok{"1 is greater than 2.}\SpecialCharTok{\textbackslash{}n}\StringTok{"}\NormalTok{)}
\NormalTok{\} }\ControlFlowTok{else}\NormalTok{ \{}
  \ControlFlowTok{if}\NormalTok{ (}\ConstantTok{FALSE}\NormalTok{) \{}
    \FunctionTok{cat}\NormalTok{(}\StringTok{"I am right.}\SpecialCharTok{\textbackslash{}n}\StringTok{"}\NormalTok{)}
\NormalTok{  \} }\ControlFlowTok{else}\NormalTok{ \{}
    \FunctionTok{cat}\NormalTok{(}\StringTok{"Don\textquotesingle{}t be stupid.}\SpecialCharTok{\textbackslash{}n}\StringTok{"}\NormalTok{)}
\NormalTok{  \}}
\NormalTok{\}}
\end{Highlighting}
\end{Shaded}

\begin{verbatim}
## Don't be stupid.
\end{verbatim}

\begin{Shaded}
\begin{Highlighting}[]
\ControlFlowTok{if}\NormalTok{ (}\ConstantTok{TRUE}\NormalTok{) }
  \FunctionTok{cat}\NormalTok{(}\StringTok{"Curly brackets are unnecessary for one{-}line statement code.}\SpecialCharTok{\textbackslash{}n}\StringTok{"}\NormalTok{)}
\end{Highlighting}
\end{Shaded}

\begin{verbatim}
## Curly brackets are unnecessary for one-line statement code.
\end{verbatim}

\subsection{While Loops}
\label{sec:BRS-while}

While loops, \textit{while()}, are used for repeated statements as long
as the condition in the parentheses () is satisfied. Curly brackets \{\}
are used to define the scope of the code. The \textit{next} statement
starts the next iteration, while the \textit{break} statement is for
exiting the loop. Nested while loops are legal in R.

\begin{Shaded}
\begin{Highlighting}[]
\NormalTok{i }\OtherTok{=} \DecValTok{1}
\ControlFlowTok{while}\NormalTok{ (i }\SpecialCharTok{\textgreater{}} \DecValTok{0}\NormalTok{) \{}
  \FunctionTok{print}\NormalTok{(}\StringTok{"i is always greater than 0, but this is not an infinite loop."}\NormalTok{)}
  \ControlFlowTok{if}\NormalTok{ (i }\SpecialCharTok{==} \DecValTok{3}\NormalTok{) \{}
    \FunctionTok{print}\NormalTok{(}\FunctionTok{paste}\NormalTok{(}\StringTok{"i equals to "}\NormalTok{, i, }\StringTok{" now. Exit loop."}\NormalTok{, }\AttributeTok{sep=}\StringTok{""}\NormalTok{))}
    \ControlFlowTok{break}
\NormalTok{  \}}
  \ControlFlowTok{if}\NormalTok{ (i }\SpecialCharTok{\textgreater{}=} \DecValTok{1}\NormalTok{) \{}
\NormalTok{    i }\OtherTok{=}\NormalTok{ i }\SpecialCharTok{+} \DecValTok{1}
    \FunctionTok{print}\NormalTok{(}\FunctionTok{paste}\NormalTok{(}\StringTok{"i equals to "}\NormalTok{, i, }\StringTok{" now."}\NormalTok{, }\AttributeTok{sep=}\StringTok{""}\NormalTok{))}
    \ControlFlowTok{next} 
\NormalTok{  \}}
\NormalTok{\}}
\end{Highlighting}
\end{Shaded}

\begin{verbatim}
## [1] "i is always greater than 0, but this is not an infinite loop."
## [1] "i equals to 2 now."
## [1] "i is always greater than 0, but this is not an infinite loop."
## [1] "i equals to 3 now."
## [1] "i is always greater than 0, but this is not an infinite loop."
## [1] "i equals to 3 now. Exit loop."
\end{verbatim}

\begin{Shaded}
\begin{Highlighting}[]
\NormalTok{x }\OtherTok{=} \DecValTok{0}
\ControlFlowTok{while}\NormalTok{ (x }\SpecialCharTok{==} \DecValTok{0}\NormalTok{)  }\CommentTok{\# curly brackets are unnecessary in a one{-}line statement}
\NormalTok{  x }\OtherTok{=}\NormalTok{ x }\SpecialCharTok{+} \DecValTok{1}
\end{Highlighting}
\end{Shaded}

\subsection{For Loops}
\label{sec:BRS-for}

For loops, \textit{for()}, iterates through a sequence. Parentheses, (),
are a must. Curly brackets \{\} are used to define the scope of the
code. The \textit{next} statement is used to jump to the next iteration,
and the \textit{break} statement is used to exit the loop. Nested for
loops are legal in R.

\begin{Shaded}
\begin{Highlighting}[]
\ControlFlowTok{for}\NormalTok{ (i }\ControlFlowTok{in} \DecValTok{1}\SpecialCharTok{:}\DecValTok{10}\NormalTok{) \{}
  \ControlFlowTok{if}\NormalTok{ (i }\SpecialCharTok{==} \DecValTok{4}\NormalTok{)}
    \ControlFlowTok{next} 
  \ControlFlowTok{else} \ControlFlowTok{if}\NormalTok{ (i }\SpecialCharTok{\textgreater{}} \DecValTok{4}\NormalTok{)}
    \ControlFlowTok{break}
  \FunctionTok{print}\NormalTok{(}\FunctionTok{paste}\NormalTok{(}\StringTok{"i equals to "}\NormalTok{, i, }\StringTok{" now."}\NormalTok{, }\AttributeTok{sep=}\StringTok{""}\NormalTok{))}
\NormalTok{\}}
\end{Highlighting}
\end{Shaded}

\begin{verbatim}
## [1] "i equals to 1 now."
## [1] "i equals to 2 now."
## [1] "i equals to 3 now."
\end{verbatim}

\begin{Shaded}
\begin{Highlighting}[]
\ControlFlowTok{for}\NormalTok{ (x }\ControlFlowTok{in} \DecValTok{1}\SpecialCharTok{:}\DecValTok{3}\NormalTok{)}
  \FunctionTok{print}\NormalTok{(}\StringTok{"Curly brackets are unnecessary in a one{-}line statement."}\NormalTok{)}
\end{Highlighting}
\end{Shaded}

\begin{verbatim}
## [1] "Curly brackets are unnecessary in a one-line statement."
## [1] "Curly brackets are unnecessary in a one-line statement."
## [1] "Curly brackets are unnecessary in a one-line statement."
\end{verbatim}

\begin{Shaded}
\begin{Highlighting}[]
\ControlFlowTok{for}\NormalTok{ (i }\ControlFlowTok{in} \DecValTok{1}\SpecialCharTok{:}\DecValTok{3}\NormalTok{) \{}
  \ControlFlowTok{for}\NormalTok{ (j }\ControlFlowTok{in} \DecValTok{1}\SpecialCharTok{:}\DecValTok{3}\NormalTok{)}
    \FunctionTok{cat}\NormalTok{(}\DecValTok{3}\SpecialCharTok{*}\NormalTok{(i}\DecValTok{{-}1}\NormalTok{)}\SpecialCharTok{+}\NormalTok{j)}
  \FunctionTok{cat}\NormalTok{(}\StringTok{"}\SpecialCharTok{\textbackslash{}n}\StringTok{"}\NormalTok{)}
\NormalTok{\}}
\end{Highlighting}
\end{Shaded}

\begin{verbatim}
## 123
## 456
## 789
\end{verbatim}

\subsection{Functions}
\label{sec:BRS-function}

Functions can be defined by keyword \textit{function()}. Curly brackets
\{\} define the scope of the code. Parameters can be specified in the
parentheses (), which those with default values must be placed last.
Values in \textit{return()} are returned.
\textbf{It is inappropriate to rely on implicit return. Always specify \textit{return()} whenever needed}.

\begin{Shaded}
\begin{Highlighting}[]
\NormalTok{myfunc }\OtherTok{=} \ControlFlowTok{function}\NormalTok{(x, }\AttributeTok{y=}\ConstantTok{FALSE}\NormalTok{) \{}
  \FunctionTok{print}\NormalTok{(}\FunctionTok{paste}\NormalTok{(}\StringTok{"x equals to "}\NormalTok{, x, }\StringTok{"."}\NormalTok{, }\AttributeTok{sep=}\StringTok{""}\NormalTok{))}
  \FunctionTok{print}\NormalTok{(}\FunctionTok{paste}\NormalTok{(}\StringTok{"The default value of y is "}\NormalTok{, y, }\StringTok{"."}\NormalTok{, }\AttributeTok{sep=}\StringTok{""}\NormalTok{))}
  \ControlFlowTok{if}\NormalTok{ (y }\SpecialCharTok{!=} \ConstantTok{FALSE}\NormalTok{) \{}
    \FunctionTok{print}\NormalTok{(}\StringTok{"x, the variable, is a parameter."}\NormalTok{)}
    \FunctionTok{print}\NormalTok{(}\FunctionTok{paste}\NormalTok{(x, }\StringTok{", the value been passed, is an argument."}\NormalTok{, }\AttributeTok{sep=}\StringTok{""}\NormalTok{))}
    \FunctionTok{print}\NormalTok{(}\FunctionTok{paste}\NormalTok{(}\StringTok{"The current argument of parameter y is "}\NormalTok{, y, }\StringTok{"."}\NormalTok{, }\AttributeTok{sep=}\StringTok{""}\NormalTok{))}
\NormalTok{  \}}
  \FunctionTok{return}\NormalTok{(}\FunctionTok{paste}\NormalTok{(}\StringTok{"myfunc() returns the value of y, "}\NormalTok{, y, }\StringTok{"."}\NormalTok{, }\AttributeTok{sep=}\StringTok{""}\NormalTok{))}
\NormalTok{\}}
\FunctionTok{myfunc}\NormalTok{(}\DecValTok{100}\NormalTok{, }\ConstantTok{TRUE}\NormalTok{)}
\end{Highlighting}
\end{Shaded}

\begin{verbatim}
## [1] "x equals to 100."
## [1] "The default value of y is TRUE."
## [1] "x, the variable, is a parameter."
## [1] "100, the value been passed, is an argument."
## [1] "The current argument of parameter y is TRUE."
\end{verbatim}

\begin{verbatim}
## [1] "myfunc() returns the value of y, TRUE."
\end{verbatim}

\begin{Shaded}
\begin{Highlighting}[]
\NormalTok{oneLineFun }\OtherTok{=} \ControlFlowTok{function}\NormalTok{(x)}
  \FunctionTok{print}\NormalTok{(x)}
\FunctionTok{oneLineFun}\NormalTok{(}\StringTok{"One{-}line statement functions without curly brackets are legal."}\NormalTok{)}
\end{Highlighting}
\end{Shaded}

\begin{verbatim}
## [1] "One-line statement functions without curly brackets are legal."
\end{verbatim}

\begin{Shaded}
\begin{Highlighting}[]
\FunctionTok{oneLineFun}\NormalTok{(}\StringTok{"But they are mostly useless."}\NormalTok{)}
\end{Highlighting}
\end{Shaded}

\begin{verbatim}
## [1] "But they are mostly useless."
\end{verbatim}

\newpage
\section{Data Type}
\label{sec:VAR}

\textit{class()} is used to check the data type of a variable. \newline
It is legal to change a data type of a variable.

\begin{Shaded}
\begin{Highlighting}[]
\NormalTok{var }\OtherTok{\textless{}{-}} \FloatTok{11.11}
\FunctionTok{class}\NormalTok{(var)}
\end{Highlighting}
\end{Shaded}

\begin{verbatim}
## [1] "numeric"
\end{verbatim}

\begin{Shaded}
\begin{Highlighting}[]
\NormalTok{var }\OtherTok{\textless{}{-}} \StringTok{"11.11"}
\FunctionTok{class}\NormalTok{(var)}
\end{Highlighting}
\end{Shaded}

\begin{verbatim}
## [1] "character"
\end{verbatim}

Data types can also be converted via \textit{as.numeric()},
\textit{as.integer()}, \textit{as.character()}, \textit{as.logical()}
and \textit{as.complex()}.

\newpage
\subsection{Numeric}
\label{sec:VAR-numeric}

Numbers are recognized as `numeric' by default, either with decimals or
not.

\begin{Shaded}
\begin{Highlighting}[]
\NormalTok{x.num }\OtherTok{\textless{}{-}} \DecValTok{11}
\FunctionTok{class}\NormalTok{(x.num)}
\end{Highlighting}
\end{Shaded}

\begin{verbatim}
## [1] "numeric"
\end{verbatim}

\begin{Shaded}
\begin{Highlighting}[]
\NormalTok{x.num }\OtherTok{\textless{}{-}} \FloatTok{11.11}
\FunctionTok{class}\NormalTok{(x.num)}
\end{Highlighting}
\end{Shaded}

\begin{verbatim}
## [1] "numeric"
\end{verbatim}

\begin{Shaded}
\begin{Highlighting}[]
\FunctionTok{as.integer}\NormalTok{(x.num)}
\end{Highlighting}
\end{Shaded}

\begin{verbatim}
## [1] 11
\end{verbatim}

\begin{Shaded}
\begin{Highlighting}[]
\FunctionTok{as.character}\NormalTok{(x.num)}
\end{Highlighting}
\end{Shaded}

\begin{verbatim}
## [1] "11.11"
\end{verbatim}

\begin{Shaded}
\begin{Highlighting}[]
\FunctionTok{as.logical}\NormalTok{(x.num)}
\end{Highlighting}
\end{Shaded}

\begin{verbatim}
## [1] TRUE
\end{verbatim}

\begin{Shaded}
\begin{Highlighting}[]
\FunctionTok{as.complex}\NormalTok{(x.num)  }\CommentTok{\# the imaginary part is 0}
\end{Highlighting}
\end{Shaded}

\begin{verbatim}
## [1] 11.11+0i
\end{verbatim}

\newpage
\subsection{Integer}
\label{sec:VAR-integer}

Numbers can be specified as `integer' by following the digits with
\textbf{'L'}.

\begin{Shaded}
\begin{Highlighting}[]
\NormalTok{x.int }\OtherTok{\textless{}{-}} \DecValTok{11}\DataTypeTok{L}  
\FunctionTok{class}\NormalTok{(x.int)}
\end{Highlighting}
\end{Shaded}

\begin{verbatim}
## [1] "integer"
\end{verbatim}

\begin{Shaded}
\begin{Highlighting}[]
\FunctionTok{as.numeric}\NormalTok{(x.int)}
\end{Highlighting}
\end{Shaded}

\begin{verbatim}
## [1] 11
\end{verbatim}

\begin{Shaded}
\begin{Highlighting}[]
\FunctionTok{as.character}\NormalTok{(x.int)}
\end{Highlighting}
\end{Shaded}

\begin{verbatim}
## [1] "11"
\end{verbatim}

\begin{Shaded}
\begin{Highlighting}[]
\FunctionTok{as.logical}\NormalTok{(x.int)}
\end{Highlighting}
\end{Shaded}

\begin{verbatim}
## [1] TRUE
\end{verbatim}

\begin{Shaded}
\begin{Highlighting}[]
\FunctionTok{as.complex}\NormalTok{(x.int)  }\CommentTok{\# the imaginary part is 0}
\end{Highlighting}
\end{Shaded}

\begin{verbatim}
## [1] 11+0i
\end{verbatim}

\newpage
\subsection{Character}
\label{sec:VAR-character}

Single line and Multiline string. Note that indentations are recognized
as parts of a multiline string.

\begin{Shaded}
\begin{Highlighting}[]
\NormalTok{str }\OtherTok{\textless{}{-}} \StringTok{"string"}
\NormalTok{str }\OtherTok{\textless{}{-}} \StringTok{"This is a }
\StringTok{String."}
\FunctionTok{print}\NormalTok{(str)    }\CommentTok{\# a \textquotesingle{}\textbackslash{}n\textquotesingle{} will be placed at each line break}
\end{Highlighting}
\end{Shaded}

\begin{verbatim}
## [1] "This is a \nString."
\end{verbatim}

\begin{Shaded}
\begin{Highlighting}[]
\FunctionTok{cat}\NormalTok{(str)      }\CommentTok{\# \textquotesingle{}\textbackslash{}n\textquotesingle{} won\textquotesingle{}t appear, but there will still be line breaks}
\end{Highlighting}
\end{Shaded}

\begin{verbatim}
## This is a 
## String.
\end{verbatim}

Calculate the number of characters in a string. Note that
\textit{length()} returns the length of a vector, not the number of
characters.

\begin{Shaded}
\begin{Highlighting}[]
\NormalTok{str }\OtherTok{\textless{}{-}} \StringTok{"Hello World"}
\FunctionTok{nchar}\NormalTok{(str)  }
\end{Highlighting}
\end{Shaded}

\begin{verbatim}
## [1] 11
\end{verbatim}

\begin{Shaded}
\begin{Highlighting}[]
\FunctionTok{length}\NormalTok{(str)}
\end{Highlighting}
\end{Shaded}

\begin{verbatim}
## [1] 1
\end{verbatim}

Check if a string contains a specific character.

\begin{Shaded}
\begin{Highlighting}[]
\FunctionTok{grepl}\NormalTok{(}\StringTok{"H"}\NormalTok{, str)}
\end{Highlighting}
\end{Shaded}

\begin{verbatim}
## [1] TRUE
\end{verbatim}

\begin{Shaded}
\begin{Highlighting}[]
\FunctionTok{grepl}\NormalTok{(}\StringTok{"A"}\NormalTok{, str)}
\end{Highlighting}
\end{Shaded}

\begin{verbatim}
## [1] FALSE
\end{verbatim}

Combine two strings.

\begin{Shaded}
\begin{Highlighting}[]
\NormalTok{str }\OtherTok{\textless{}{-}} \FunctionTok{paste}\NormalTok{(}\StringTok{"Hello"}\NormalTok{, }\StringTok{"World"}\NormalTok{)}
\end{Highlighting}
\end{Shaded}

Escape characters. Note that using \textit{print()} or auto-printing
will print out the backslash. Use \textit{cat()} to show the intended
string.

\begin{Shaded}
\begin{Highlighting}[]
\NormalTok{text }\OtherTok{\textless{}{-}} \StringTok{"I love }\SpecialCharTok{\textbackslash{}"}\StringTok{you}\SpecialCharTok{\textbackslash{}"}\StringTok{"}
\FunctionTok{print}\NormalTok{(text)}
\end{Highlighting}
\end{Shaded}

\begin{verbatim}
## [1] "I love \"you\""
\end{verbatim}

\begin{Shaded}
\begin{Highlighting}[]
\FunctionTok{cat}\NormalTok{(text)}
\end{Highlighting}
\end{Shaded}

\begin{verbatim}
## I love "you"
\end{verbatim}

\begin{Shaded}
\begin{Highlighting}[]
\FunctionTok{cat}\NormalTok{(}\StringTok{"I love }\SpecialCharTok{\textbackslash{}\textbackslash{}}\StringTok{ you"}\NormalTok{, }\StringTok{"}\SpecialCharTok{\textbackslash{}n}\StringTok{"}\NormalTok{)  }\CommentTok{\# \textquotesingle{}\textbackslash{}\textbackslash{}\textquotesingle{} a backslash}
\end{Highlighting}
\end{Shaded}

\begin{verbatim}
## I love \ you
\end{verbatim}

\begin{Shaded}
\begin{Highlighting}[]
\FunctionTok{cat}\NormalTok{(}\StringTok{"I love }\SpecialCharTok{\textbackslash{}r}\StringTok{ you"}\NormalTok{, }\StringTok{"}\SpecialCharTok{\textbackslash{}n}\StringTok{"}\NormalTok{)  }\CommentTok{\# \textquotesingle{}\textbackslash{}r\textquotesingle{} covers the previous output}
\end{Highlighting}
\end{Shaded}

\begin{verbatim}
## I love  you
\end{verbatim}

\begin{Shaded}
\begin{Highlighting}[]
\FunctionTok{cat}\NormalTok{(}\StringTok{"I love }\SpecialCharTok{\textbackslash{}t}\StringTok{ you"}\NormalTok{, }\StringTok{"}\SpecialCharTok{\textbackslash{}n}\StringTok{"}\NormalTok{)  }\CommentTok{\# \textquotesingle{}\textbackslash{}t\textquotesingle{} add a tab}
\end{Highlighting}
\end{Shaded}

\begin{verbatim}
## I love    you
\end{verbatim}

\begin{Shaded}
\begin{Highlighting}[]
\CommentTok{\#cat("I love \textbackslash{}b you", "\textbackslash{}n")  \# \textquotesingle{}\textbackslash{}b\textquotesingle{} backspace}
\end{Highlighting}
\end{Shaded}

\begin{Shaded}
\begin{Highlighting}[]
\NormalTok{x.chr }\OtherTok{\textless{}{-}} \StringTok{"11.11"}
\FunctionTok{class}\NormalTok{(x.chr)}
\end{Highlighting}
\end{Shaded}

\begin{verbatim}
## [1] "character"
\end{verbatim}

\begin{Shaded}
\begin{Highlighting}[]
\FunctionTok{as.numeric}\NormalTok{(x.chr)}
\end{Highlighting}
\end{Shaded}

\begin{verbatim}
## [1] 11.11
\end{verbatim}

\begin{Shaded}
\begin{Highlighting}[]
\FunctionTok{as.integer}\NormalTok{(x.chr)}
\end{Highlighting}
\end{Shaded}

\begin{verbatim}
## [1] 11
\end{verbatim}

\begin{Shaded}
\begin{Highlighting}[]
\FunctionTok{as.logical}\NormalTok{(x.chr)  }\CommentTok{\# not TRUE or FALSE}
\end{Highlighting}
\end{Shaded}

\begin{verbatim}
## [1] NA
\end{verbatim}

\begin{Shaded}
\begin{Highlighting}[]
\FunctionTok{as.complex}\NormalTok{(x.chr)}
\end{Highlighting}
\end{Shaded}

\begin{verbatim}
## [1] 11.11+0i
\end{verbatim}

If characters are included, the conversion leads to `NA'.

\begin{Shaded}
\begin{Highlighting}[]
\NormalTok{x.chr }\OtherTok{\textless{}{-}} \StringTok{"11.11+1i"}
\FunctionTok{as.numeric}\NormalTok{(x.chr)  }
\end{Highlighting}
\end{Shaded}

\begin{verbatim}
## Warning: NAs introduced by coercion
\end{verbatim}

\begin{verbatim}
## [1] NA
\end{verbatim}

\begin{Shaded}
\begin{Highlighting}[]
\FunctionTok{as.integer}\NormalTok{(x.chr)}
\end{Highlighting}
\end{Shaded}

\begin{verbatim}
## Warning: NAs introduced by coercion
\end{verbatim}

\begin{verbatim}
## [1] NA
\end{verbatim}

\begin{Shaded}
\begin{Highlighting}[]
\FunctionTok{as.logical}\NormalTok{(x.chr)}
\end{Highlighting}
\end{Shaded}

\begin{verbatim}
## [1] NA
\end{verbatim}

\begin{Shaded}
\begin{Highlighting}[]
\FunctionTok{as.complex}\NormalTok{(x.chr)      }\CommentTok{\# conversion will work only in this form}
\end{Highlighting}
\end{Shaded}

\begin{verbatim}
## [1] 11.11+1i
\end{verbatim}

\begin{Shaded}
\begin{Highlighting}[]
\NormalTok{x.chr }\OtherTok{\textless{}{-}} \StringTok{"11.11+ 1i"}   
\FunctionTok{as.complex}\NormalTok{(x.chr)      }\CommentTok{\# space is not ignored}
\end{Highlighting}
\end{Shaded}

\begin{verbatim}
## Warning: NAs introduced by coercion
\end{verbatim}

\begin{verbatim}
## [1] NA
\end{verbatim}

\newpage
\subsection{Logical}
\label{sec:VAR-logical}

\textbf{TRUE} is stored as `1' and \textbf{FALSE} is stored as `0'. For
numeric, integer and complex, it is recognized as \textbf{TRUE} unless
the value is 0. Characters can't be converted into logical.

\begin{Shaded}
\begin{Highlighting}[]
\NormalTok{x.log }\OtherTok{\textless{}{-}} \ConstantTok{TRUE}
\FunctionTok{class}\NormalTok{(x.log)}
\end{Highlighting}
\end{Shaded}

\begin{verbatim}
## [1] "logical"
\end{verbatim}

\begin{Shaded}
\begin{Highlighting}[]
\FunctionTok{as.numeric}\NormalTok{(x.log)}
\end{Highlighting}
\end{Shaded}

\begin{verbatim}
## [1] 1
\end{verbatim}

\begin{Shaded}
\begin{Highlighting}[]
\FunctionTok{as.integer}\NormalTok{(x.log)}
\end{Highlighting}
\end{Shaded}

\begin{verbatim}
## [1] 1
\end{verbatim}

\begin{Shaded}
\begin{Highlighting}[]
\FunctionTok{as.character}\NormalTok{(x.log)}
\end{Highlighting}
\end{Shaded}

\begin{verbatim}
## [1] "TRUE"
\end{verbatim}

\begin{Shaded}
\begin{Highlighting}[]
\FunctionTok{as.complex}\NormalTok{(x.log)  }\CommentTok{\# imaginary part is 0}
\end{Highlighting}
\end{Shaded}

\begin{verbatim}
## [1] 1+0i
\end{verbatim}

\textbf{TRUE} and \textbf{FALSE} are returned by comparisons done by
comparison operators. Note that multiple comparison (eg.
\(x == y == z\)) is \textbf{illegal} in R.

\begin{Shaded}
\begin{Highlighting}[]
\DecValTok{1} \SpecialCharTok{==} \DecValTok{2}   \CommentTok{\# equal}
\end{Highlighting}
\end{Shaded}

\begin{verbatim}
## [1] FALSE
\end{verbatim}

\begin{Shaded}
\begin{Highlighting}[]
\DecValTok{1} \SpecialCharTok{!=} \DecValTok{2}   \CommentTok{\# not equal}
\end{Highlighting}
\end{Shaded}

\begin{verbatim}
## [1] TRUE
\end{verbatim}

\begin{Shaded}
\begin{Highlighting}[]
\DecValTok{1} \SpecialCharTok{\textgreater{}}  \DecValTok{2}   \CommentTok{\# greater than}
\end{Highlighting}
\end{Shaded}

\begin{verbatim}
## [1] FALSE
\end{verbatim}

\begin{Shaded}
\begin{Highlighting}[]
\DecValTok{1} \SpecialCharTok{\textless{}}  \DecValTok{2}   \CommentTok{\# less than}
\end{Highlighting}
\end{Shaded}

\begin{verbatim}
## [1] TRUE
\end{verbatim}

\begin{Shaded}
\begin{Highlighting}[]
\DecValTok{1} \SpecialCharTok{\textgreater{}=} \DecValTok{2}   \CommentTok{\# not less than}
\end{Highlighting}
\end{Shaded}

\begin{verbatim}
## [1] FALSE
\end{verbatim}

\begin{Shaded}
\begin{Highlighting}[]
\DecValTok{1} \SpecialCharTok{\textless{}=} \DecValTok{2}   \CommentTok{\# not greater than}
\end{Highlighting}
\end{Shaded}

\begin{verbatim}
## [1] TRUE
\end{verbatim}

Logical operators can combine conditional statements. \textbf{\&} and
\textbf{|} are \textbf{element-wise operators}, which they can compare
each element in an object (such as vector, array, matrix, etc.). Note
that logical comparison are processed \textbf{from right to left}.

\begin{Shaded}
\begin{Highlighting}[]
\ConstantTok{TRUE} \SpecialCharTok{\&}  \ConstantTok{FALSE} \SpecialCharTok{|} \ConstantTok{FALSE}
\end{Highlighting}
\end{Shaded}

\begin{verbatim}
## [1] FALSE
\end{verbatim}

\begin{Shaded}
\begin{Highlighting}[]
\ConstantTok{TRUE} \SpecialCharTok{|}  \ConstantTok{FALSE} \SpecialCharTok{\&} \ConstantTok{FALSE}   
\end{Highlighting}
\end{Shaded}

\begin{verbatim}
## [1] TRUE
\end{verbatim}

\begin{Shaded}
\begin{Highlighting}[]
\ConstantTok{TRUE} \SpecialCharTok{||} \ConstantTok{FALSE}           \CommentTok{\# or}
\end{Highlighting}
\end{Shaded}

\begin{verbatim}
## [1] TRUE
\end{verbatim}

\begin{Shaded}
\begin{Highlighting}[]
\ConstantTok{TRUE} \SpecialCharTok{\&\&} \ConstantTok{FALSE}           \CommentTok{\# and}
\end{Highlighting}
\end{Shaded}

\begin{verbatim}
## [1] FALSE
\end{verbatim}

\begin{Shaded}
\begin{Highlighting}[]
\SpecialCharTok{!}\DecValTok{1}                      \CommentTok{\# not}
\end{Highlighting}
\end{Shaded}

\begin{verbatim}
## [1] FALSE
\end{verbatim}

\newpage
\subsection{Complex}
\label{sec:VAR-complex}

\begin{Shaded}
\begin{Highlighting}[]
\NormalTok{x.cpx }\OtherTok{\textless{}{-}} \DecValTok{1}\DataTypeTok{i}\SpecialCharTok{+}\DecValTok{2}  \CommentTok{\# \textquotesingle{}i\textquotesingle{} is the imaginary part}
\FunctionTok{class}\NormalTok{(x.cpx)}
\end{Highlighting}
\end{Shaded}

\begin{verbatim}
## [1] "complex"
\end{verbatim}

\begin{Shaded}
\begin{Highlighting}[]
\NormalTok{x }\OtherTok{\textless{}{-}} \DecValTok{2}\SpecialCharTok{+}\DecValTok{1}\DataTypeTok{i}
\FunctionTok{class}\NormalTok{(x.cpx)}
\end{Highlighting}
\end{Shaded}

\begin{verbatim}
## [1] "complex"
\end{verbatim}

\begin{Shaded}
\begin{Highlighting}[]
\FunctionTok{as.integer}\NormalTok{(x.cpx)  }\CommentTok{\# the imaginary part is dropped}
\end{Highlighting}
\end{Shaded}

\begin{verbatim}
## Warning: imaginary parts discarded in coercion
\end{verbatim}

\begin{verbatim}
## [1] 2
\end{verbatim}

\begin{Shaded}
\begin{Highlighting}[]
\FunctionTok{as.numeric}\NormalTok{(x.cpx)  }\CommentTok{\# the imaginary part is dropped}
\end{Highlighting}
\end{Shaded}

\begin{verbatim}
## Warning: imaginary parts discarded in coercion
\end{verbatim}

\begin{verbatim}
## [1] 2
\end{verbatim}

\begin{Shaded}
\begin{Highlighting}[]
\FunctionTok{as.character}\NormalTok{(x.cpx)}
\end{Highlighting}
\end{Shaded}

\begin{verbatim}
## [1] "2+1i"
\end{verbatim}

\begin{Shaded}
\begin{Highlighting}[]
\FunctionTok{as.logical}\NormalTok{(x.cpx)}
\end{Highlighting}
\end{Shaded}

\begin{verbatim}
## [1] TRUE
\end{verbatim}

\newpage
\section{Data Structure}
\label{sec:DS}

\begin{Shaded}
\begin{Highlighting}[]
\DecValTok{1}\SpecialCharTok{:}\DecValTok{10}
\end{Highlighting}
\end{Shaded}

\begin{verbatim}
##  [1]  1  2  3  4  5  6  7  8  9 10
\end{verbatim}

\begin{Shaded}
\begin{Highlighting}[]
\DecValTok{1} \SpecialCharTok{\%in\%} \DecValTok{1}\SpecialCharTok{:}\DecValTok{10}
\end{Highlighting}
\end{Shaded}

\begin{verbatim}
## [1] TRUE
\end{verbatim}

\begin{Shaded}
\begin{Highlighting}[]
\CommentTok{\# \%*\% matrix multiplication}
\end{Highlighting}
\end{Shaded}

\newpage
\section{Figure}
\label{sec:Fig}

\newpage
\section{Statistical Analysis}
\label{sec:SA}

\newpage
\section{String Manipulation}
\label{sec:SM}

\newpage
\section{Important Packages}
\label{sec:Pkg}

\newpage
\section{Misc}
\label{sec:Misc}

\end{document}
